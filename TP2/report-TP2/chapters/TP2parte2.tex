\chapter{Parte II}

\section{Análise do problema}

Na segunda parte é proposto ao grupo simular a gestão da empresa W\&W por um período de 200 dias, recorrendo ao software "Jogo da Distribuição", facultado com o enunciado do trabalho. O objetivo será maximizar o lucro total durante o período especificado, utilizando a política de encomenda determinada na parte anterior.

\section{Resultados obtidos}

\subsection{Nível de stock}

As seguintes tabelas apresentam a evolução do nível de stock a cada 40 dias de operação: 

\input{./report-TP2/table/tabelaStockA}

\input{./report-TP2/table/tabelaStock1}

\begin{table}[htpb]
\begin{center}
\begin{tabular}{cc}
\toprule
Dia & Stock	       \\ \midrule
40 & 6                 \\ 
80 & 1                 \\ 
120 & 13               \\ 
160 & 30               \\ 
200 & 16     	       \\ 
\bottomrule
\end{tabular}
\end{center}
\caption{Nível de stock na loja 2}
\label{tab:tabela3}
\end{table}



\begin{table}[htpb]
\begin{center}
\begin{tabular}{cc}
\toprule
Dia & Stock	 	\\ \midrule
40 & 17                 \\ 
80 & 16                 \\ 
120 & 6                 \\ 
160 & 14                \\ 
200 & 9     		\\ 
\bottomrule
\end{tabular}
\end{center}
\caption{Nível de stock na loja 3}
\label{tab:tabela4}
\end{table}




\subsection{Saldo Acumulado}

No gráfico abaixo está representada a evoulução do saldo acumulado durante os 200 dias de jogo:

\begin{figure}[h]
	\centering
	\includegraphics[scale=0.75]{./report/img/saldo.png}
	\caption{Saldo acumulado no final do jogo}
\label{fig:figure1}
\end{figure}


\newpage

\subsection{Estratégia de jogo}

No dia inicial o stock do armazém foi inicializado em 125 unidades, e o stock de cada uma das lojas a 25 unidades. A política de encomendas seguida tanto para as lojas como para o armazém corresponde á política determinada na parte I para cada entidade:

\begin{itemize}
	\iteam \emph{Política de encomendas para o armazém}
		\begin{itemize}
			\item \emph{q} = 276 unidades;
			\item \emph{S} = 118 unidades;
			\item Frequência de encomenda: 37 dias;
		\end{itemize}
	\iteam \emph{Política de encomendas para as lojas}
		\begin{itemize}
			\item \emph{q} = 17 unidades;
			\item \emph{S} = 5 unidades;
			\item Frequência de encomenda: 7 dias;
		\end{itemize}
\end{itemize}

A política de Nível de encomenda foi seguida á risca durante o jogo. Tanto para o armazém como para as lojas, sempre que o nível de inventário descia abaixo de \emph{S}, ou o número de dias desde o último pedido igualava o valor de frequência de encomenda, era simulada a encomenda de \emph{q} unidades de artigo para a entidade correspondente. Terminados os 200 dias, foi obtido um saldo final de aproximadamente 40705 euro. O uso dos resultados obtidos na Parte I permitiram atingir facilmente um saldo final considerável, o que comprova a utilidade dos processos matemáticos utilizados para modelar a gestão do nível de inventário.




