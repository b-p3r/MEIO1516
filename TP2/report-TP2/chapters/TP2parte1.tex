\chapter{Parte I}

\section{Análise do problema}

O objetivo da primeira parte corresponde a modelar a gestão de inventário para a empresa fictícia W\&W, através da aplicação da política de gestão de inventários do tipo Nível de Encomenda. Esta empresa armazena o stock comprado ao fabricante no seu armazém central, que por sua vez o distribui por três lojas para venda ao público. O armazém e as lojas operam de forma diferente, facto que requere a criação de um modelo diferente para cada tipo de entidade.

\subsection{Pressupostos assumidos} 

Considerando que a falta de stock numa loja resulta na perda do lucro associado á venda do artigo, ou seja, a diferença entre o preço de venda ao cliente final e o preço de aquisição ao fabricante. Consequentemente, foi assumido que o custo de quebra nas lojas é igual á margem de lucro perdida na venda de cada artigo.

A margem de lucro em cada artigo vendido é também o custo de quebra assumido para o armazém, pois este distribui o stock da empresa pelas três lojas de venda ao público. A quebra de stock no armazém irá inevitavelmente interromper o fornecimento de artigo ás lojas, causando a perda de vendas, e do lucro associado. 

A taxa de procura para o armazém não está especificada no enunciado. No entanto, a taxa de procura diária de cada uma das três lojas cujo stock provem do armazém segue uma distribuição uniforme, entre 0 a 5 unidades. Considerando este facto, é plausível assumir que a procura diária para o armazém é igual á soma da procura nas três lojas de que o armazém fornece.  

É também conveniente considerar que os artigos comercializados por a W\&W não requerem nenhuma atenção especial do ponto de vista logístico. Num caso real, as propriedades inerentes a cada artigo podem levantar dificuldades operacionais, ou até mesmo tornar certas políticas de gestão inventário impraticáveis. 

Por fim, apesar de todas as distribuições usadas na parte I do trabalho serem uniformes, assumimos, segundo as indicações do enunciado, que todas as variáveis \emph{DDLT} seguem leis Normais. 


\subsection{Cálculos}

Para modelar a gestão de inventário da empresa W\&W, será necessário determinar a quantidade de encomenda \emph{q}, e o nível de inventário \emph{S}. No enunciado são providenciados os seguintes dados:

\begin{itemize}
	\item \emph{Dados relativos ao armazém:}
		\begin{itemize}
			\item Valor de aquisição por artigo: 70 euros;
			\item Custo por encomenda á fábrica: 200 euros;
			\item Tempo de entrega da fábrica: 10 + (D1/2) dias;
			\item Taxa de posse anual: 21\%;
		\end{itemize}
	\item \emph{Dados relativos ás lojas:}
		\begin{itemize}
			\item Procura: Distribuição uniforme, entre 0 a 5 unidades;
			\item Preço unitário de venda: 100 euros;
			\item Custo por entrega do armazém: 2.75 euros;
			\item Tempo de entrega do armazém: 3 dias;
			\item Taxa de posse anual: 25\%;
		\end{itemize}
\end{itemize}

\subsection{Política Nível de Encomenda para o armazém}

O armazém é responsável por a gestão do nível de inventário da empresa, servindo de intermediário entre o fabricante do artigo e as lojas de venda ao público. Utilizando os dados anteriormente referidos determinamos:

\begin{itemize}
\item Prazo de entrega \emph{l} = 10 + \dfrac{8}{2} = 14  dias;
\item Custo anual de posse \emph{C1} = \emph{b} \times \emph{i} = 70 \times 0.21 = 14.70 euros por artigo por ano;
\item Custo de quebra \emph{C2} = 100 - 70 = 30 euros por artigo;
\item Custo de passagem de encomenda \emph{C3} = 200 euros por encomenda;
\end{itemize}

Para o cálculo do prazo de entrega de uma encomenda ao armazém, foi usado o último dígito do número de aluno 72628.

Tal como anteriormente referido, assume-se para o armazém uma procura uniformemente distribuída, entre 0 e 15 unidades, que corresponde á soma das procuras individuais de cada loja por este abastecida. Desta forma, é utilizado:

\begin{itemize}
\item Distribuição \emph{X} \approx Uniforme[0;15];
\item Média da distribuição: \dfrac{15 - 0}{2} = 7.5 unidades;
\item Desvio padrão: \sqrt{18.75} = 4.3301;
\end{itemize}

Inicializando o processo de cálculo com \emph{E(DDLT > S)} = 0:

\begin{itemize}
\item \emph{1ª iteração:}
 
	\emph{q*}=\sqrt{\dfrac{2 \times r \times (C2 \times E(DDLT > S) + C3)}{C1}}
	\emph{q*}=\sqrt{\dfrac{2 \times 7.5 \times 365 \times (30 \times 0 + 200)}{14.70}}
	\emph{q*}=272.9281 \approx 273 unidades;

	Com \emph{q} determinado, é possível calcular \emph{P(DDLT > S)} com:

	\emph{P(DDLT > S)}=\dfrac{\emph{C1 \times q}}{\emph{C2 \times r}}
	\emph{P(DDLT > S)}=\dfrac{14.70 \times 273}{30 \times 7.5 \times 365}
	\emph{P(DDLT > S)}=0.0489;

	Através da tabela \emph{Área da Distribuição Normal Standard, N(0,1)}, disponível nos apontamentos da unidade curricular, obtem-se \emph{Z} \approx 1.66;
	
	Para determinar o segundo integral, necessário ao calculo de \emph{E(DDLT > S)}, temos:
	
	\emph{Z}=\dfrac{3 \times \emph{N}}{100}
	\emph{N}=55;

	Através da tabela \emph{Função de Densidade Normal Standard, N(0,1)}, disponível nos apontamentos da unidade curricular, obtem-se o segundo integral: 0.018440;

	Com o segundo integral, é possível calcular \emph{E(DDLT > S)}, utilizando:

	\emph{E(DDLT > S)} = \emph{segundo integral} \times \emph{\sigma}\emph{DDLT}
	\emph{E(DDLT > S)} = 0.018440 \times \sqrt{emph{l} \times \emph{sigma_r^2}}
	\emph{E(DDLT > S)} = 0.018440 \times \sqrt{14 \times 4.3301}
	\emph{E(DDLT > S)} = 0.018440 \times 7.7860
	\emph{E(DDLT > S)} = 0.1436;

	\emph{S} = \emph{\mu}\emph{DDLT} + \emph{Z} \times \emph{\sigma}\emph{DDLT}
	\emph{S} = \emph{r} \times \emph{l} + \emph{Z} \times \emph{\sigma}\emph{DDLT}
	\emph{S} = 7.5 \times 14 + 1.66 \times 7.7860
	\emph{S} = 117.9248 \approx 118 unidades;

	
\item \emph{2ª iteração:}
	
	Para a segunda iteração é utilizado \emph{E(DDLT > S)} = 0.1436;
	
	\emph{q*}=\sqrt{\dfrac{2 \times r \times (C2 \times E(DDLT > S) + C3)}{C1}}
	\emph{q*}=\sqrt{\dfrac{2 \times 7.5 \times 365 \times (30 \times 0.1436 + 200)}{14.70}}
	\emph{q*}=275.8519 \approx 276 unidades;

	Com \emph{q} determinado, é possível calcular \emph{P(DDLT > S)} utilizando:

	\emph{P(DDLT > S)}=\dfrac{\emph{C1 \times q}}{\emph{C2 \times r}}
	\emph{P(DDLT > S)}=\dfrac{14.70 \times 276}{30 \times 7.5 \times 365}
	\emph{P(DDLT > S)}=0.0494;

	Através da tabela \emph{Área da Distribuição Normal Standard, N(0,1)}, disponível nos apontamentos da unidade curricular, obtem-se \emph{Z} \approx 1.65;
	
	Para determinar o segundo integral, necessário ao calculo de \emph{E(DDLT > S)}, temos:
	
	\emph{Z}=\dfrac{3 \times \emph{N}}{100}
	\emph{N}=55;

	Através da tabela \emph{Função de Densidade Normal Standard, N(0,1)}, disponível nos apontamentos da unidade curricular, obtem-se o segundo integral: 0.018440;

	Com o segundo integral, é possível calcular \emph{E(DDLT > S)}, com:

	\emph{E(DDLT > S)} = 0.018440 \times \emph{\sigma}\emph{DDLT}
	\emph{E(DDLT > S)} = 0.018440 \times \sqrt{emph{l} \times \emph{sigma_r^2}}
	\emph{E(DDLT > S)} = 0.018440 \times \sqrt{14 \times 4.3301}
	\emph{E(DDLT > S)} = 0.018440 \times 7.7860
	\emph{E(DDLT > S)} = 0.1436;

	Como o valor de \emph{E(DDLT > S)} da iteração atual é igual ao valor da iteração anterior, esta é a última iteração.

	\emph{S} = \emph{\mu}\emph{DDLT} + \emph{Z} \times \emph{\sigma}\emph{DDLT}
	\emph{S} = \emph{r} \times \emph{l} + \emph{Z} \times \emph{\sigma}\emph{DDLT}
	\emph{S} = 7.5 \times 14 + 1.65 \times 7.7860
	\emph{S} = 117.8469 \approx 118 unidades;

\end{itemize}

De acordo com os cálculos efetudos, determinamos para o armazém os seguintes valores:

\begin{itemize}
\item Quantidade de encomenda \emph{q}: 276 unidades;
\item Nível de inventário \emph{S}: 118 unidades;
\end{itemize}

Os resultados obtidos permitem concluir que uma ordem de encomenda com a quantia constante de 276 unidades deverá ser enviada sempre que o stock em armazém atinge valores inferiores a 118 artigos. Adicionalmente, e sendo a procura conhecida, é possível calcular o valor aproximado de:

\begin{itemize}
\item Frequência de encomendas: \dfrac{\emph{q}}{\emph{r}} = \dfrac{276}{7.5} \approx 37 dias;
\item Encomendas anuais: \dfrac{\emph{r}}{\emph{q}} = \dfrac{7.5 \times 365}{276} \approx 10 encomendas;
\end{itemize}

Ao seguir esta política, o armazém realizará por ano cerca de 10 encomendas, com um intervalo entre dois pedidos de encomenda consecutivos aproximadamente igual a 37 dias.   


\subsection{Política Nível de Encomenda para as lojas}

Para cada uma das lojas, são especificados os seguintes dados:

\begin{itemize}
\item Custo anual de posse \emph{C1} = \emph{b} \times \emph{i} = 70 \times 0.25 = 17.50 euros por artigo por ano;
\item Custo de quebra \emph{C2} = 30 euros por artigo;
\item Custo de passagem de encomenda \emph{C3} = 2.75 euros por encomenda;
\item Procura 
\end{itemize}

A procura em cada uma das lojas segue uma distribuição uniforme, entre 0 a 5 unidades de artigo. Consequentemente, temos:

\begin{itemize}
\item Distribuição \emph{X} \approx Uniforme[0;5];
\item Média da distribuição: \dfrac{5 - 0}{2} = 2.5 unidades
\item Variância: \dfrac{(5 - 0)^2}{12} = 2.0833;
\item Desvio padrão: \sqrt{2.0833} = 1.4434;
\end{itemize}

Inicializando o processo de cálculo com \emph{E(DDLT > S)} = 0:

\begin{itemize}
\item \emph{1ª iteração:}

	\emph{q*}=\sqrt{\dfrac{2 \times r \times (C2 \times E(DDLT > S) + C3)}{C1}}
	\emph{q*}=\sqrt{\dfrac{2 \times 2.5 \times 365 \times (30 \times 0 + 2.75)}{17.50}}
	\emph{q*}=16.9347 \approx 17 unidades;

	Com \emph{q} determinado, é possível calcular \emph{P(DDLT > S)} utilizando:

	\emph{P(DDLT > S)}=\dfrac{\emph{C1 \times q}}{\emph{C1 \times q}+\emph{C2 \times r}}
	\emph{P(DDLT > S)}=\dfrac{\emph{17.50 \times 17}}{\emph{17.50 \times 17}+\emph{30 \times 2.5 \times 365}}
	\emph{P(DDLT > S)}=0.0108;

	Utilizando o valor de \emph{P(DDLT > S)}, é possível determinar \emph{S} com:
	
	\emph{P(DDLT > S)}=\int_S^5 \mathrm{p(n)}\,\mathrm{d}x
	0.0108=\int_S^5 \mathrm{1/5}\,\mathrm{d}x
	0.0108=\dfrac{5}{5} - \dfarc{S}{5}
	\emph{S}=4.946 unidades;

	Com \emph{S} calculado, determinamos \emph{E(DDLT > S)} utilizando:
	
	\emph{E(DDLT > S)}=\int_S^5 \mathrm{xp(n)}\,\mathrm{d}x - emph{S} \times \emph{P(DDLT > S)}
	\emph{E(DDLT > S)}=\dfrac{25}{10} - \dfrac{24.4630}{10} - 4.946 \times 0.0108
	\emph{E(DDLT > S)} = 0.0003;


\item \emph{2ª iteração:}
	
	Para a segunda iteração é utilizado \emph{E(DDLT > S)} = 0.0003;

	\emph{q*}=\sqrt{\dfrac{2 \times r \times (C2 \times E(DDLT > S) + C3)}{C1}}
	\emph{q*}=\sqrt{\dfrac{2 \times 2.5 \times 365 \times (30 \times 0.0003 + 2.75)}{17.50}}
	\emph{q*}=16.9624 \approx 17 unidades

	Com \emph{q} determinado, é possível calcular \emph{P(DDLT > S)} utilizando:

	\emph{P(DDLT > S)}=\dfrac{\emph{C1 \times q}}{\emph{C1 \times q}+\emph{C2 \times r}}
	\emph{P(DDLT > S)}=\dfrac{\emph{17.50 \times 17}}{\emph{17.50 \times 17}+\emph{30 \times 2.5 \times 365}}
	\emph{P(DDLT > S)}=0.0108;

	Utilizando o valor de \emph{P(DDLT > S)}, é possível determinar \emph{S} com:
	
	\emph{P(DDLT > S)}=\int_S^5 \mathrm{p(n)}\,\mathrm{d}x
	0.0108=\int_S^5 \mathrm{1/5}\,\mathrm{d}x
	0.0108=\dfrac{5}{5} - \dfarc{S}{5}
	\emph{S}=4.946 unidades;

	Com \emph{S} calculado, determinamos \emph{E(DDLT > S)} utilizando:
	
	\emph{E(DDLT > S)}=\int_S^5 \mathrm{xp(n)}\,\mathrm{d}x - emph{S} \times \emph{P(DDLT > S)}
	\emph{E(DDLT > S)}=\dfrac{25}{10} - \dfrac{24.4630}{10} - 4.946 \times 0.0108
	\emph{E(DDLT > S)} = 0.0003

	O valor de \emph{E(DDLT > S)} da interação atual é igual ao valor da iteração anterior, logo, esta é a última iteração.

\end{itemize}

De acordo com os cálculos efetudos, determinamos para cada uma das lojas os seguintes valores:

\begin{itemize}
\item Quantidade de encomenda \emph{q}: 17 unidades;
\item Nível de inventário \emph{S}: 5 unidades;
\end{itemize}

Os resultados obtidos permitem concluir que uma ordem de reabastecimento com a quantia constante de 17 unidades deverá ser enviada sempre que o stock em loja atinge valores inferiores a 5 artigos. Adicionalmente, e sendo a procura conhecida, é possível calcular o valor aproximado de:

\begin{itemize}
\item Frequência de encomendas: \dfrac{\emph{q}}{\emph{r}} = \dfrac{17}{2.5} \approx 7 dias;
\item Encomendas anuais: \dfrac{\emph{r}}{\emph{q}} = \dfrac{2.5 \times 365}{17} \approx 54 encomendas;
\end{itemize}

Ao seguir esta política, cada loja lançará por ano cerca de 54 pedidos de reabastecimento, com um intervalo entre dois pedidos consecutivos aproximadamente igual a 7 dias.   


