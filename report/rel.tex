\documentclass[pdftex,12pt,a4paper]{report}

\usepackage[dvipsnames]{xcolor}
\usepackage[pdftex]{graphicx}
\usepackage{float}
\usepackage{fancyvrb}
\fvset{xleftmargin=2em}

\usepackage{pgfplots}
\pgfplotsset{width=10cm,compat=1.9}
\usepackage{tikzscale}
\usepackage{pgfplotstable}
\usepackage{booktabs}
\usepackage[font=small,labelfont=bf,tableposition=top]{caption}

\usepackage[utf8]{inputenc}
\usepackage[portuges]{babel}
\usepackage[T1]{fontenc}
\usepackage{times}
%\usepackage{lmodern}
\usepackage[obeyspaces,spaces]{url}
\usepackage[left=20mm,right=20mm,top=25mm,bottom=25mm]{geometry}
\usepackage{titlesec}
\usepackage{mathtools}
\usepackage{amsfonts}

%identa 1º paragrafo de capitulos e secções
\usepackage{indentfirst}
\usepackage{url}
%\usepackage{alltt}
\usepackage[]{hyperref}
\usepackage{xspace}
\usepackage[final]{pdfpages}

\hypersetup{
%pdftitle={Trabalho 1 - Gestão de Projeto},
%pdfauthor={Bruno Pereira},
%pdfsubject={Investigação Operacional},
%pdfkeywords={keyword1, keyword2}},
bookmarksnumbered=true,     
bookmarksopen=true,         
bookmarksopenlevel=1,       
colorlinks=true,            
pdfstartview=Fit,           
pdfpagemode=UseOutlines, % this is the option you were lookin for
pdfpagelayout=TwoPageRight
		}

\usepackage{minted}
\usemintedstyle{borland}
\setminted{frame=lines,
framesep=2mm,
baselinestretch=1.2,
fontsize=\footnotesize,
linenos, 
breaklines,
breakautoindent=false,
autogobble
}
\usepackage{caption}

\begin{document}

\begin{titlepage}
\includepdf[pages={1}]{./report/front.pdf}


\end{titlepage}






\begin{abstract}





\end{abstract}

\tableofcontents
\input{./report/chapters/intro}

\chapter{Parte 1}
\label{cap:p1}




\section{Dimensionamento do serviço de clientes}

\subsection{Análise do problema}
\subsubsection{Dados}
\subsubsection{Cálculos}





\newpage
\subsection{Testes e Resultados}


\section{Pressupostos considerados}






\chapter{Parte 2}
\label{cap:p2}

\section{Resumo de artigo}







\input{./report/chapters/concl}


\appendix

\chapter{Código do Programa}




\begin{verbatim}

public class Aula()

  {

    int n, m;

    int max(int a, int b)

      {

       ......

       return(max);

      }

  }

\end{verbatim}



\begin{verbatim}

llll sanjdb c kjnjcnjnjj mmmmmmmmmmmmm hhhhhhhhhhhhhhhhhhhhhhhh
jjjjjjjjjjjjjjjjjjjjjjjjjjjj kkkkkkkkkkkkkkkkkk

      aqui deve aparecer o código do programa,

      tal como está formato no ficheiro-fonte "darius.java"

      caso indesejável $\varepsilon$

\end{verbatim}



\begin{minted}{c}

#include <stdio.h>
#define N 10
/* Block
 * comment */

 int main()
 {
     int i;
	 
	   // Line comment.
		 puts("Hello world!");
			     
		 for (i = 0; i < N; i++)
		 {
		 puts("LaTeX is also great for programmers!");
		 }
										 
	   return 0;
				
}
\end{minted}


E ainda possível importar diretamente o ficheiro:


%\begin{longlisting}
%	\inputminted{c}{Exercicio2/exe2_1.l}
%	\caption{Ficheiro fonte do exercício 2.1}
%	\label{listing:3}
%\end{longlisting}





\bibliographystyle{alpha}

\bibliography{./report/bibs/pl}

\end {document}


